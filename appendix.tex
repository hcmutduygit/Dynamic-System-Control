\chapter{Code matlab}
    \section{Biểu đồ Bode}
    \begin{lstlisting}[caption={Code vẽ biểu đồ Bode}, label={lst:bode}]
    % Delete all variables and close all figures
    clear all;
    clc;
    close all;

    % Define the transfer function G(s)
    num = [4.85];
    den = [1, 0, 53.51];
    G = tf(num, den);
    bode(G)
    \end{lstlisting}
    \section{Poles và Zeros}
    \begin{lstlisting}[caption={Code vẽ poles và zeros}, label={lst:pz}]
    % Delete all variables and close all figures
    clear all;
    clc;
    close all;
    
    % Define the transfer function G(s)
    num = [4.85];        % Numerator (4.85)
    den = [1, 0, 53.51]; % Denominator (s^2 + 0*s + 53.51)
    G1 = tf(num, den);   % Create transfer function G1(s)
    
    % Create the closed-loop system G_cl(s)
    G_cl = feedback(G1, 1);   % The close_loop system: G_cl(s) = G1(s)/(1 + G1(s))
    
    % Plot the poles and zeros of the closed-loop system
    figure;
    pzmap(G_cl);              % Plot
    title('Pole-zero diagram of closed loop system G_1(s)');
    grid on;
    set(gca, 'FontSize', 12);
    
    % Display poles and zeros
    [poles, zeros] = pzmap(G_cl);  % Get poles and zeros
    disp('The poles of closed-loop system:');
    disp(poles);
    disp('The zeros of closed-loop system:');
    disp(zeros);
    \end{lstlisting}
    \section{Đáp ứng hệ thống chưa điều khiển}
    \begin{lstlisting}[caption={Code vẽ đáp ứng hệ thống chưa điều khiển}, label={lst:step}]
    % Clear variables and previous plots
    clear all;
    clc;
    close all;
    
    % Define the transfer function G1(s)
    num = [4.85];              % Numerator
    den = [1, 0, 53.51];       % Denominator (s^2 + 53.51)
    G1 = tf(num, den);         % Create transfer function G1(s)
    
    % Create closed-loop system with unit feedback
    G_cl = feedback(G1, 1);    % Closed-loop system: G_cl(s) = G1(s)/(1 + G1(s))
    
    % 1. Plot step response
    figure;
    step(G_cl, 10);            % Plot step response over 10 seconds
    title('Step Response of Closed-Loop System G_1(s)');
    xlabel('Time (s)');
    ylabel('Angle \phi (rad)');
    grid on;
    set(gca, 'FontSize', 12);
    
    % 2. Plot impulse response
    figure;
    impulse(G_cl, 10);         % Plot impulse response over 10 seconds
    title('Impulse Response of Closed-Loop System G_1(s)');
    xlabel('Time (s)');
    ylabel('Angle \phi (rad)');
    grid on;
    set(gca, 'FontSize', 12);
    
    % 3. Compute performance metrics (overshoot, settling time)
    % Use stepinfo to extract step response characteristics
    info = stepinfo(G_cl, 'SettlingTimeThreshold', 0.02); % 2% threshold for settling time
    
    % Display performance metrics
    disp('Step Response Information:');
    disp(['Overshoot (%): ', num2str(info.Overshoot)]);
    disp(['Settling Time (s): ', num2str(info.SettlingTime)]);
    
    % 4. Compute final value (theoretical steady-state value)
    final_value = dcgain(G_cl);  % Final value = DC gain of the closed-loop system
    disp(['Final Value (Theoretical): ', num2str(final_value)]);
    \end{lstlisting}



    
